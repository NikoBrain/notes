\section{Consequences}
\lecture{05}{Aug 22 10:15}{Consequences}

\subsection{Properties of Zero}

\begin{proposition}[Zero Property]\label{prop:zero-multiplication}
    Let $F$ be a field. Then $\forall a \in F$, $a \cdot 0 = 0$.
\end{proposition}

\begin{proof}
    Let $a \in F$. Then:
    \begin{align*}
    a \cdot \tilde{0} &= a \cdot \tilde{0} + \tilde{0} \\
    &= a \cdot \tilde{0} + \left(a \cdot \tilde{0} + \left(-\left(a \cdot \tilde{0}\right)\right)\right) = \left(a \cdot \tilde{0} + a \cdot \tilde{0}\right) + \left(-\left(a \cdot \tilde{0}\right)\right) \\
    &= a \cdot (\tilde{0} + \tilde{0}) + \left(-\left(a \cdot \tilde{0}\right)\right) = a \cdot \tilde{0} + \left(-\left(a \cdot \tilde{0}\right)\right) = \tilde{0}
    \end{align*}
\end{proof}
\begin{proposition}[Zero Product Property]\label{prop:zero-product}
    Let $F$ be a field. Then $\forall a, b \in F$:
    \[
        a \cdot b = 0 \implies a = 0 \text{ or } b = 0
    \]
\end{proposition}

\begin{proof}
    Either $a = 0$ or $a \neq 0$.
    
    If $a = 0$, then the conclusion is satisfied.
    
    If $a \neq 0$, then by field axiom (vii), $\exists a^{-1} \in F$ such that $a \cdot a^{-1} = a^{-1} \cdot a = 1$. Therefore:
    \begin{align*}
        b &= b \cdot 1 = b \cdot (a \cdot a^{-1}) = (b \cdot a) \cdot a^{-1} = 0 \cdot a^{-1} = 0
    \end{align*}
\end{proof}

\begin{proposition}[Zero Non-Invertible]\label{prop:zero-non-invertible}
    Let $F$ be a field. Then $0 \in F$ is not invertible with respect to multiplication.
\end{proposition}

\begin{proof}
    Suppose, for contradiction, that $\exists a \in F$ such that $a \cdot 0 = 1$.
    
    Then by Proposition 1.2.1, $1 = a \cdot 0 = 0$.
    
    But by field axiom (viii), $0 \neq 1$, which is a contradiction.
\end{proof}

\subsection{Modular Arithmetic}

The integers modulo $m$ is the set $\Z_m = \{0, 1, 2, \ldots, m-1\}$ with addition and multiplication defined by taking remainders after division by $m$.

\begin{eg}
    $\Z_2 = \{0, 1\}$ with operation tables:
    \begin{center}
    \begin{tabular}{c|cc} 
        $+$ & 0 & 1 \\
        \hline
        0 & 0 & 1 \\
        1 & 1 & 0
    \end{tabular}
    \qquad
    \begin{tabular}{c|cc}
        $\cdot$ & 0 & 1 \\
        \hline
        0 & 0 & 0 \\
        1 & 0 & 1
    \end{tabular}
    \end{center}
    This is a field.
\end{eg}

\begin{eg}
    $\Z_3 = \{0, 1, 2\}$ with operation tables:
    \begin{center}
    \begin{tabular}{c|ccc}
        $+$ & 0 & 1 & 2 \\
        \hline
        0 & 0 & 1 & 2 \\
        1 & 1 & 2 & 0 \\
        2 & 2 & 0 & 1
    \end{tabular}
    \qquad
    \begin{tabular}{c|ccc}
        $\cdot$ & 0 & 1 & 2 \\
        \hline
        0 & 0 & 0 & 0 \\
        1 & 0 & 1 & 2 \\
        2 & 0 & 2 & 1
    \end{tabular}
    \end{center}
    This is a field.
\end{eg}

\begin{eg}
    $\Z_4 = \{0, 1, 2, 3\}$ is not a field because $2 \cdot 2 = 0$ but $2 \neq 0$, violating the zero product property.
\end{eg}

\begin{note}
    \begin{itemize} These facts are important in applications but not central to linear algebra
        \item When $m$ is prime, $\Z_m$ is a field
        \item When $m$ is not prime, $\Z_m$ is not a field (by the zero product property) (doesn't mean no field with $m$ members isn't possible)
    \end{itemize}
\end{note}

\subsection{Operations Involving Inverses}

\begin{proposition}[Inverse Properties]\label{prop:inverse-properties}
    Let $F$ be a field and let $a, b \in F$. Then:
    \begin{enumerate}[label=(\roman*)]
        \item $-0 = 0$
        \item $1^{-1} = 1$
        \item $-(-a) = a$
        \item $(a^{-1})^{-1} = a$ (when $a \neq 0$)
        \item $(-1) \cdot a = -a$
        \item $(-a) \cdot b = -(a \cdot b)$
        \item $(-a) \cdot b = a \cdot (-b)$
        \item $(-a) \cdot (-b) = a \cdot b$
    \end{enumerate}
\end{proposition}

\begin{proof}\
    \begin{enumerate}[label=(\roman*)]
        \item Since $0 + 0 = 0$, we have $-0 = 0$ by uniqueness of additive inverses.
        
        \item Since $1 \cdot 1 = 1$, we have $1^{-1} = 1$ by uniqueness of multiplicative inverses.
        
        \item Since $(-a) + a = 0$, we have $-(-a) = a$ by uniqueness of additive inverses.
        
        \item Since $a^{-1} \cdot a = 1$, we have $(a^{-1})^{-1} = a$ by uniqueness of multiplicative inverses.
        
        \item $a + ((-1) \cdot a) = 1 \cdot a + ((-1) \cdot a) = (1 + (-1)) \cdot a = 0 \cdot a = 0$
        
        \item $(-a) \cdot b = ((-1) \cdot a) \cdot b = (-1) \cdot (a \cdot b) = -(a \cdot b)$
        
        \item $(-a) \cdot b = ((-1) \cdot a) \cdot b = (-1) \cdot (a \cdot b) = a \cdot ((-1) \cdot b) = a \cdot (-b)$
        
        \item $(-a) \cdot (-b) = ((-1) \cdot a) \cdot ((-1) \cdot b) = (-1) \cdot (-1) \cdot a \cdot b$
        
        Since $(-1) \cdot (-1) = (-1)^{-1} \cdot (-1) = 1$, we get:
        $(-a) \cdot (-b) = 1 \cdot a \cdot b = a \cdot b$
    \end{enumerate}
\end{proof}