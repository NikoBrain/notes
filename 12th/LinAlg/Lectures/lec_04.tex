\setcounter{chapter}{0}
\chapter{Fields}
\section{Definition}
\lecture{04}{Aug 25 10:15}{Definition}

\subsection{Algebra}

Algebra is the study of equations and calculations. Calculations are done by carrying out operations on objects of an \textbf{algebraic structure}.

\begin{definition*}[Algebraic Structure]
    A set of objects with operations defined on it.
\end{definition*}

\subsection{Numbers as Models}

Numbers are an algebraic structure constructed to model certain types of physical objects and their interactions:
\begin{itemize}
    \item Objects such as apples and skittles
    \item Interactions such as trading skittles for apples
\end{itemize}

Interactions in the natural world have certain characteristics, discovered by observation. Operations on numbers have properties matching those observations.


\subsection{The Natural Numbers Model}

The natural numbers $\N$ consist of:
\begin{itemize}
    \item A set $\N$ of objects, called numbers
    \item An algorithm assigning each number a unique name
    \item An algorithm for deciding which number is "next"
    \item A binary operation called addition, denoted by $+$
    \item An algorithm for finding $a + b$ for any $a, b \in \N$
\end{itemize}

These algorithms are constructed to have properties modeled after observed characteristics of combining physical objects.

\subsection{The Integer Numbers Model}

Additional observations reveal interactions not adequately modeled by $\N$. The integers $\Z$ extend $\N$ by:
\begin{itemize}
    \item Adding a new number, called zero and denoted $0$
    \item Matching each number $a \in \N$ with a new number $-a$
    \item Extending the addition algorithm for these new numbers
\end{itemize}

\subsection{Abstraction and Generalization}

Exploration in mathematics proceeds by abstraction and generalization:
\begin{enumerate}
    \item Start with a familiar mathematical model whose properties are understood
    \item Investigate a hypothetical, abstract structure defined by those properties
    \item Any claim proven about the abstract structure is true for any concrete model satisfying the same axioms
\end{enumerate}

\subsection{Groups}

\begin{definition}[Group]\label{def:group}
    A group is a set $G$ equipped with an operation $\star$ that satisfies:
    \begin{enumerate}[label=(\roman*)]
        \item $\forall a, b \in G$, $a \star b \in G$ (closure)
        \item $\forall a, b, c \in G$, $(a \star b) \star c = a \star (b \star c)$ (associativity)
        \item $\exists e \in G$ such that $\forall a \in G$, $a \star e = e \star a = a$ (neutral element)
        \item $\forall a \in G$, $\exists b \in G$ such that $a \star b = b \star a = e$ (invertibility)
    \end{enumerate}
\end{definition}

\begin{definition}[Commutative Group]\label{def:commutative-group}
    A commutative group is a group that also satisfies $\forall a, b \in G$, $a \star b = b \star a$.
\end{definition}

\begin{eg}\
    \begin{itemize}
        \item $(\Z, +)$ is a group
        \item $(\Z, \cdot)$ is not a group (fails invertibility)
        \item Hours on a clock with "passage of time" operation forms a group
    \end{itemize}
\end{eg}

Groups are important because many different things can be modeled as groups, but they don't capture everything we need for linear algebra.

\subsection{The Rational Numbers Model}

Additional observations lead to the need for division. The rational numbers are:
\[
    \Q = \left\{\frac{a}{b} : a, b \in \Z, b \neq 0\right\}
\]

We can represent rational numbers as ordered pairs $(a,b)$ where:
\begin{itemize}
    \item Equality: $(a,b) = (c,d) \iff ad = bc$
    \item Addition: $(a,b) + (c,d) = (ad + bc, bd)$
    \item Multiplication: $(a,b) \cdot (c,d) = (ac, bd)$
\end{itemize}

\subsection{Properties of \Q}

From the definition of operations on $\Q$:
\begin{enumerate}[label=(\roman*)]
    \item $\Q$ is closed with respect to both addition and multiplication
    \item Both operations are associative
    \item Both operations are commutative
    \item $(0,1) \in \Q$ is neutral with respect to addition, $(1,1) \in \Q$ is neutral with respect to multiplication
    \item Multiplication is distributive over addition: $a(b + c) = ab + ac$
    \item Every $q \in \Q$ is invertible with respect to addition; every $q \neq (0,1)$ is invertible with respect to multiplication
\end{enumerate}

\subsection{Fields}

\begin{definition}[Field]\label{def:field}
    A field is a set $F$ equipped with two binary operations, called addition and multiplication and denoted $+$ and $\cdot$ respectively, that satisfy:
    \begin{enumerate}[label=(\roman*)]
        \item $\forall a, b \in F$: $a + b \in F$ and $a \cdot b \in F$ (closure)
        \item $\forall a, b, c \in F$: $(a + b) + c = a + (b + c)$ and $(a \cdot b) \cdot c = a \cdot (b \cdot c)$ (associativity)
        \item $\forall a, b \in F$: $a + b = b + a$ and $a \cdot b = b \cdot a$ (commutativity)
        \item $\forall a, b, c \in F$: $a \cdot (b + c) = (a \cdot b) + (a \cdot c)$ (distributivity)
        \item $\exists \tilde{0} \in F$ and $\exists \tilde{1} \in F$ such that $\forall a \in F$: $a + \tilde{0} = a$ and $a \cdot \tilde{1} = a$ (neutral elements)
        \item $\forall a \in F$, $\exists a' \in F$ such that $a + a' = \tilde{0}$ (additive inverses)
        \item $\forall a \neq \tilde{0} \in F$, $\exists a' \in F$ such that $a \cdot a' = \tilde{1}$ (multiplicative inverses)
        \item $\tilde{0} \neq \tilde{1}$ (distinct neutral elements)
    \end{enumerate}
\end{definition}

\begin{note} The two operations are not completely symmetric
    \begin{itemize}
        \item Multiplication distributes over addition, but not vice versa
        \item The additive neutral is not required to have a multiplicative inverse
    \end{itemize}
\end{note}

\subsection{Notation and Terminology}

\begin{itemize}
    \item Elements of a field are called \textbf{scalars}
    \item We drop the tildes and refer to $\tilde{0}$ and $\tilde{1}$ as "zero" and "one"
    \item Multiplication precedes addition in order of operations
    \item We often omit the multiplication symbol: $ab + ac$ instead of $a \cdot b + a \cdot c$
\end{itemize}

\subsection{Uniqueness of Inverses}

\begin{proposition}[Uniqueness of Additive Inverse]\label{prop:unique-additive-inverse}
    Let $F$ be a field and $a \in F$. There exists a unique $a' \in F$ such that $a + a' = 0$.
\end{proposition}

\begin{proof}
    Let $a \in F$ and suppose $a', a'' \in F$ such that
    \begin{align*}
        a' + a &= a + a' = 0 \\
        a'' + a &= a + a'' = 0
    \end{align*}
    
    Then:
    \begin{align*}
        a' &= a' + 0 = a' + (a + a'') = (a' + a) + a'' = 0 + a'' = a''
    \end{align*}
\end{proof}

\begin{proposition}[Uniqueness of Multiplicative Inverse]\label{prop:unique-multiplicative-inverse}
    Let $F$ be a field and $a \neq 0 \in F$. There exists a unique $a' \in F$ such that $a \cdot a' = 1$.
\end{proposition}

\begin{proof}
    Let $a \in F$ and suppose $a', a'' \in F$ such that
    \begin{align*}
        a' \cdot a &= a \cdot a' = 1 \\
        a'' \cdot a &= a \cdot a'' = 1
    \end{align*}
    
    Then:
    \begin{align*}
        a' &= a' \cdot 1 = a' \cdot (a \cdot a'') = (a' \cdot a) \cdot a'' = 1 \cdot a'' = a''
    \end{align*}
\end{proof}

\subsection{Inverse Notation and Operations}

Since inverses are unique:
\begin{itemize}
    \item The additive inverse of $a$ is denoted $(-a)$
    \item The multiplicative inverse of $a$ is denoted $a^{-1}$
    \item Subtraction: $a - b \coloneqq a + (-b)$
    \item Division: when $b \neq 0$, $a/b \coloneqq a \cdot b^{-1}$
\end{itemize}
