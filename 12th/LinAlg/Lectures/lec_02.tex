\section{Proofs}
\lecture{02}{Aug 19 12:41}{Proofs}
\subsection{Deductive Reasoning}
\begin{definition*}[Deductive Reasoning]\label{def:deductive-reasoning}
    The process of making deductive arguments
\end{definition*}
\begin{definition*}[Deductive Argument]\label{def:deductive-argument}
    Process of making a logical inference
\end{definition*}
\begin{definition*}[Inference]\label{def:inference}
    Claim that a certain \hyperref[def:predicate]{predicate}, called the \textbf{conclusion}, follows from one or more \hyperref[def:predicate]{predicates}, called the \textbf{premises}.
\end{definition*}
Predicate $B$ \textbf{follows from} $A$ if it is impossible simultaneously for $A$ to be True and $B$ to be False

An inference is \textbf{valid} if the conclusion follows from the premises

A deductive argument is \textbf{sound} if the inference is valid and its premises are True

\subsection{More Logic Notation and Terminology}
\begin{definition*}[Universal Quantification Symbol ($\forall$)]\label{def:universal-quantification-symbol}
    Denotes that a proposition is True for all members
\end{definition*}
\begin{eg}
    \[
    \forall x \in \N:x^2 \geq x
    \]
    Read: ``\textbf{for all} (every, any, each) $x$ in \N\ the predicate ($x^2 \geq x$) evaluates to True''
\end{eg}
\begin{definition*}[Existential Quantification Symbol($\exists$)]\label{def:existential-quantification-symbol}
    Denotes that a proposition is True for at least one member
\end{definition*}
\begin{eg}
    \[
    \exists x \in \N:x^2<x
    \]
    Read: ``\textbf{there exists} an $x$ in \N\ such that the predicate $x^2<x$ evaluates to true''

    
\end{eg}
\begin{note}
        This is an example of a False proposition
\end{note}

\subsection{Contraposition}
\begin{definition*}[Contraposition]\label{def:contraposition}
    
    Let $p,q$ be predicates and consider the conditional statement
    \[
    p \implies q
    \]
    The \textbf{contrapositive} form of the statement is
    \[
    \neg q \implies \neg p
    \]

    A conditional statement and its contrapositive form are \textbf{equivalent}
    \[
    (p \implies q) \iff (\neg q \implies \neg p)
    \]
    When judging truthfulness of a statement, it sometimes helps to consider its contrapositive
\end{definition*}
Also existing, but not as common are:
\begin{itemize}
    \item The \textbf{inverse}: $\neg p \implies \neg q$
    \item The \textbf{converse}: $q \implies p$
    \item The \textbf{complement}: $\neg(p \implies q)$
\end{itemize}

\subsection{Proofs}

\begin{definition*}[Mathematical Proof]\label{def:proof}
    A deductive argument about something related to math. Uses spken/written language, or even sketches/diagrams.

    Usually rigorous (spells out assumptions and deductive steps as is convenient) but informal (some natural language with occasionally ambiguous symbols/rules) deductive reasoning
\end{definition*}
\begin{proposition}
    Let $n,m \in \N$ and suppose $n,m$ are even; then $(n+m)$ is even
\end{proposition}
\begin{proof}\ 
\begin{itemize}
    \item Since $n$ is even $\exists k \in \N$ such that $n=2k$
    \item Since $m$ is even $\exists q \in \N$ such that $n=2q$
    \item Then $(n+m)=(2k+2q)=2(k+q)$
    \item Therefore, $(n+m)$ is even
\end{itemize}
\begin{note}
    We used a method called \textbf{direct proof}
\end{note}
\end{proof}
\subsection{Mathematical Induction}
\begin{definition*}[Mathematical Induction]\label{def:mathematical-induction}
    If asked to prove that a certain proposition, $P(n)$ is true for any $n \in \N$, we will accept as proof the following inference, if sound:
    \begin{equation}
        \begin{array}{c}
            P(1) = \text{T} \tag{1} \\
            \left(P(k) \stackrel{2}{=} \text{T}\right) \Rightarrow \left(P(k+1) = \text{T}\right)
        \end{array}
        \stackrel{3}{\Rightarrow} P(n) = \text{T}, \forall n \in \mathbb{N}
    \end{equation}

    Where (1) is called the \textbf{base case}; (2) is called the \textbf{induction hypothesis}; (3) is the \textbf{induction step}
\end{definition*}
% A variant called \textbf{complete} (or strong/generalized) induction uses a stronger hypothesis in the induction step:
% \[
% (P(j),\forall j \leq k) \implies P(k+1)
% \]
