\chapter{The Scientific Method}
\section{Steps to the Scientific Method}
\begin{enumerate}
    \item Make an Observation
    \item Ask a question
    \item Experiment: test the hypothesis and gather data
    \item Analyze the data
    \item Draw a conclusion
\end{enumerate}
\section{Characteristics of a Good Experiment}
\begin{itemize}
    \item Tests \underline{one variable at a time}: If more than one thing is tested at a time, it won't be clear which variable caused the end result
    \item \underline{Fair and unbiased}: Experimenter must not allow his or her opinions to influence the experiment
    \item \underline{Repeated trails}: Repeating the trials in the experiment will reduce the effect of experimental errors and give a more accurate conclusion
\end{itemize}
\section{Variables}
\begin{definition}[Variable]\label{def:variable}
    A variable is anything in an experiment that can change or vary
    \begin{itemize}
        \item Any factors that can have an effect on the outcome of the experiment
    \end{itemize}
    There are three main types of variables:
    \begin{definition}[Independent Variable]\label{def:independent_variable}
        The variable intentionally changed by the scientist
        \begin{itemize}
            \item What is tested or manipulated
            \item Only change on independent variable at a time
        \end{itemize}
    \end{definition}
    \begin{definition}[Dependent Variable (Responding Variable)]\label{def:dependent_variable}
        Something that is affected by the change in the independent variable
        \begin{itemize}
            \item What is observed and measured (Data collected)
        \end{itemize}
    \end{definition}
    \begin{definition}[Controlled Variable]\label{def:controlled_variable}
        Variables that are not changed, constants
    \end{definition}
\end{definition}
\section{Control Group}
\begin{definition}[Control Group]\label{def:control_group}
    Group that isn't tested, but used for comparison as a reference for what "normal would be like
    \begin{definition}[Positive Control]\label{def:positive_control}
        Group that you expect to give a positive result
    \end{definition}
    \begin{definition}[Negative Control]\label{def:negative_control}
        Group that you expect to give a negative result
    \end{definition}
    Both ensure the validity of the experiment
\end{definition}
\section{Hypothesis}
\begin{definition}[Hypothesis]\label{def:hypothesis}
    Proposed explanation for a set of observations
    \begin{itemize}
        \item Leads to predictions that can be tested in experiments
        \item Should be based off past experiments and background research
        \item Not only a prediction, not a research question, not a theory
        \item "If... then... because..."
    \end{itemize}
\end{definition}