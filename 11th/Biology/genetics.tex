\chapter{Genetics}
\newcommand{\pangenesisdef}{Proposed around 400 BCE by Hippocrates, was an early explanation for inheritance that suggested that particles called pangenes came from all parts of the organism to be incorporated into eggs or sperm and characterists acquired during the parents' lifetime could be transfered to offspring.}
\begin{definition}[Pangenesis] \label{definition:pangenesis}
    \pangenesisdef
\end{definition}
 Aristotle rejected pangenesis; instead of particles, the potential to produce the traits was inherited

 \newcommand{\blendinghdef}{Hereditary materials mix in forming offspring.}
 \begin{definition}[Blending Hypothesis] \label{definition:blendingh}
    \blendinghdef

    Suggested in the 19th century by scientists studying plants; later rejected because it did not explain how traits that disappear in one generation can reappear in later generations.
 \end{definition}

\newcommand{\hereditydef}{The transmission of traits from one generation to the next}
\begin{definition}[Heredity] \label{definition:heredity}
    \hereditydef
\end{definition}
\newcommand{\geneticsdef}{the scientific study of heredity}
\begin{definition}[Genetics] \label{definition:genetics}
    \geneticsdef
\end{definition}
\newcommand{\mendeldef}{began the field of genetics in the 1860s, deducing the principles of genetics by breeding garden peas, relying upon a background of mathematics, physics, and chemistry.}
\begin{definition}[Gergor Mendel] \label{definition:mendel}
    \mendeldef

    In 1866, Mendel correctly argued that parents pass on to their offspring discrete ``heritable factors'' and stressed that the heritable factors (today called genes), retain their individuality generation after generation.
\end{definition}

\newcommand{\characterdef}{A heritable feature that varies among individuals, such as flower color}
\begin{definition}[Character] \label{definition:character}
    \characterdef
\end{definition}
\newcommand{\traitdef}{Each variable character}
\begin{definition}[Trait] \label{definition:trait}
    \traitdef
\end{definition}

\newcommand{\truebreedingdef}{True-breeding varieties result when self-fertilization produces offspring all identical to the parent}
\begin{definition}[True-breeding] \label{definition:truebreeding}
    \truebreedingdef
\end{definition}
\newcommand{\hybriddef}{The offspring of two different varieties}
\begin{definition}[Hybrid] \label{definition:hybrid}
    \hybriddef

    The cross-fertilization is a hybridization, or genetic cross
\end{definition}
\newcommand{\pgendef}{True-breeding parental plants}
\begin{definition}[P generation] \label{definition:pgen}
    \pgendef
\end{definition}

\newcommand{\fodef}{Hybrid offspring}
\begin{definition}[F\textsubscript{1}] \label{definition:F1}
    \fodef
\end{definition}
\newcommand{\ftdef}{A cross of F1 plants}
\begin{definition}[F\textsubscript{2}] \label{definition:f2}
    \ftdef
\end{definition}

\newcommand{\monohybriddef}{A cross between two individuals differing in a single character}
\begin{definition}[Monohybrid Cross] \label{definition:monohybrid}
    \monohybriddef
    
    Mendel performed a monohybrid cross between a plant with purple flowers and a plant with white flowers.

    The F\textsubscript{1} generation produced all plants with purple flowers
    The F\textsubscript{2} generation produced three plants with purple flowers and one plant with white flowers
\end{definition}

Mendel developed four hypotheses:
\newcommand{\alelesdef}{Alleles are alternative versions of genes that account for variations in inherited characters}
\begin{definition}[Alleles] \label{definition:aleles}
    \alelesdef
\end{definition}

For each characteristic, an organism inherits two alleles, one from each parent. The alleles can be the same or different
\newcommand{\homozygousdef}{Identical alleles}
\begin{definition}[Homozygous Genotype] \label{definition:homozygous}
    \homozygousdef
\end{definition}
\newcommand{\heterozygousdef}{Two different alleles}
\begin{definition}[Heterozygous Genotype] \label{definition:heterozygous}
    \heterozygousdef
\end{definition}


If the alleles of an inherited pair differ, then one determines the organism's appearance and is called the dominant allele. The other has no noticeable effect on the organism's appearance and is called the recessive allele.


\newcommand{\phenotypedef}{The appearance or expression of a trait.}
\begin{definition}[Phenotype] \label{definition:phenotype}
    \phenotypedef
\end{definition}

\newcommand{\genotypedef}{The genetic makup of a trait.}
\begin{definition}[Genotype] \label{definition:genotype}
    \genotypedef
\end{definition}

\newcommand{\lawofsegdef}{A sperm or egg carries only one allele for each inherited character because allele pairs separate (segregate) from each other during the production of gametes.}
\begin{definition}[Law of Segregation] \label{definition:lawofseg}
    \lawofsegdef
\end{definition}

\newcommand{\locusdef}{The specific location of a gene along a chromosome}
\begin{definition}[locus] \label{definition:locus}
    \locusdef

    For a pair of homologous chromosomes, the alleles of a gene reside at the same locus (Homozygous have same allele on both, Heterozygous have different allele on both homologues)
\end{definition}

\newcommand{\dihybriddef}{Mating of parental varieties that differ in two characters}
\begin{definition}[Dihybrid Cross] \label{definition:dihybrid}
    \dihybriddef
    P generation: RRYY and rryy
    F1 generation: all RrYy
    F2 generation:
    \begin{itemize}
        \item 9/16 with R and Y dominant
        \item 4/16 with r and Y dominant
        \item 4/16 with R and y dominant
        \item 1/16 with r and y dominant
    \end{itemize}
\end{definition}

\newcommand{\testcrossdef}{The mating between an individual of unknown genotype and a homozygous recessive individual}
\begin{definition}[Testcross] \label{definition:testcross}
    \testcrossdef

    Can show whether the unknown genotype includes a recessive allele
\end{definition}