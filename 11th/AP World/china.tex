\chapter{China}\label{ch:china}
\section*{East Asia, 960-1275 CE}
\begin{wrapfigure}[8]{r}{0.5\textwidth}
    \begin{tikzpicture}
        \coordinate (A) at (-5,0) {};
        \coordinate (B) at ( 5,0) {};
        \coordinate (C) at (0,5.2) {};
        \draw[name path=AC] (A) -- (C);
        \draw[name path=BC] (B) -- (C);
        \foreach \y/\A in {
            0/Merchants:\\ Low status because they get rich from others' work.,
            1/Peasants:\\ Worked the land for food and shelter,
            2/Nobles:\\ Gave loyalty and army to the king,
            3/King:\\ Gave land \\to nobles in\\ exchange of loyalty,} {
            \path[name path=horiz] (A|-0,\y) -- (B|-0,\y);
            \draw[name intersections={of=AC and horiz,by=P},
                name intersections={of=BC and horiz,by=Q}] (P) -- (Q)
                node[midway,above,align=center] {\A};
        }
    \end{tikzpicture}
\end{wrapfigure}

\section{Geography of China}
Self-Sufficient:
\begin{itemize}
    \item Yellow and Yangtze Rivers
    \item Fertile soil in the North and Southeast regions
\end{itemize}
Isolated Geography:
\begin{itemize}
    \item West: Plateau of Tibet and Himalayas
    \item North: Gobi Desert
    \item East: Pacific
\end{itemize}
Well connected through trade:
\begin{itemize}
    \item \href{ch:silkroads}{Silk Roads}
    \item Ocean
\end{itemize}
\section{Chinese Economy}
Trend: Feudalism (System of Obligation) endured throughout the economy

Vast majority were peasants
\begin{itemize}
    \item Main source of taxes and labor for public works
    \item No serfdom meant more freedom, they can leave or marry
\end{itemize}

\section{Chinese Ideologies}
Trend: Confucian philosophy would dominate politics; however, Daoism and Buddhism was popular with the people too

\newcommand{\mandatedef}{Gave a divine right to rule. }
\begin{concept}[Mandate of Heaven] \label{concept:mandate}
    \mandatedef
    Used by Chinese rulers to legitimize power
\end{concept}

\newcommand{\daoismdef}{School believing the key to eliminating Chaos was to live in the present and find harmony with nature. }
\begin{ideology}[Daoism (School of Way, Dao)] \label{ideology:daoism}
    \daoismdef
    \newcommand{\laozidef}{(c.570-530 BCE) Founder of Daoism, a legendary official.}
    \begin{person}[Laozi] \label{person:laozi}
        \laozidef
    \end{person}
    \begin{itemize}
        \item Man should adjust his internal balance to the rhythm of the natural world.
        \item Man should depart from society regularly to meditate and communicate with nature
    \end{itemize}
    \newcommand{\yinyangdef}{Opposing natural forces that represent the duality of nature nad the universe. }
    \begin{concept}[Yin and Yang] \label{concept:yinyang}
        \yinyangdef
        Must accept that it is always there.
    \end{concept}
\end{ideology}

\newcommand{\confucianismdef}{Stability occurs if society is organized based on the 5 relationships of superiors and inferiors}
\begin{ideology}[Confucianism] \label{ideology:confucianism}
    \confucianismdef
    \begin{itemize}
        \item Ruler and Subject
        \item Father and Son
        \item Husband and Wife
        \item Older Brother and Younger Brother
        \item Older Friend and Younger Friend
    \end{itemize}
    \newcommand{\confuciusdef}{(551-479 BCE) Founder of Confucianism}
    \begin{person}[Confucius] \label{person:confucius}
        \confuciusdef
    \end{person}
    \newcommand{\filialdef}{There is an obligation to family and to respect for ancestors, leaders, parents, and the older generation.}
    \begin{concept}[Filial Piety] \label{concept:filial}
        \filialdef
    \end{concept}
    \newcommand{\analectsdef}{Compilation of the teachings of Confucious}
    \begin{concept}[Analects] \label{concept:analects}
        \analectsdef
    \end{concept}
    Emperors used Confucianism to create loyal officials and to justify and maintain their rule
    \begin{itemize}
        \item Promotes Subject to Ruler relationship
        \item Officials were trained to obey the emperor
    \end{itemize}
\end{ideology}
\newcommand{\csexamdef}{One must master confucian philosohpy and adminstrative skills. }
\begin{concept}[Civil Service Exam (CSE)] \label{concept:csexam}
    \csexamdef
    Created a meritocracy. All Chinese males (but merchants) could take it. Favored the upper class due to resource requirements.
\end{concept}

\pdfmarkupcomment[markup=Underline, color=blue]{Previous}{None} \hyperref[ch:song]{Next}